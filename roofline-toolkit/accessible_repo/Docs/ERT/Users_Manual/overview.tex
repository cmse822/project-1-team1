\subsection{Overview}

The Roofline Performance Model for computational performance has been shown
to be a useful simplification of the complexity of processor and memory
systems.  An overview of this model can be found at:

https:\+//crd.lbl.gov/\+departments/\+computer-science/\+performance-and-al\+go\+rith\+ms-research/\+research/\+roofline

This site also contains pointers to available software, such as this tool, and
references to publications pertaining to the Roofline Performance Model.

The Roofline Toolkit was created to help software developers use the Roofline
Performance Model to improve the performance of their codes.  One of the
requirements for using the Roofline Performance Model is having a roofline
characterization of the target machine.

In the past, this characterization was obtained from the machine
specification data.  This was time consuming and error prone.  In addition, it
lead to theoretical maximums that might not be achievable by any code running
on the machine.

The Empirical Roofline Tool, ERT, generates the required characterization of a
machine empirically.  It does this by running kernel codes on the machine so
the results are, by definition, attainable by some code(s) and the options
used to compile and run the code are known.  The ERT generates the bandwidth
and gflop/sec data needed by the Roofline Performance Model.

The input to the ERT is a configuration file and the final output is a
roofline graph in PostScript format and the roofline parameters in JSON
format.  There is also a significant amount of intermediate data generated
that can be explored and used to better understand a given kernel code and
machine.

Currently, the ERT can utilize MPI, OpenMP, and Cuda/GPU for parallelization.

\subsection{Installation}

To install ERT, you simply need to use ``git'' to ``clone'' the Roofline
Toolkit located at https:\+//bitbucket.org/\+berkeleylab/\+cs-roofline-toolkit
on your local machine.

The ERT is under ``Empirical\_Roofline\_Tool-1.1.0''.  There you will find a
``README'' file, a PDF of this document (``ERT\_Users\_Manual.pdf''), the ERT
executable (``ert''), some sample configuration files
(``Config/config.ert.*''), and everything else used by the ERT.

To complete the installation, you need to make sure you have a few other
pieces of software installed:

\begin{itemize}

\vspace{-0.1in}
\item{Python that is at least version 2.6 and below version 3.x}

\vspace{-0.1in}
\item{GNUplot that is at least version 4.2.x}

\end{itemize}

To run in parallel, using MPI, OpenMP, and/or Cuda/GPU, you will need:

\begin{itemize}

\vspace{-0.1in}
\item{A working installation of MPI}

\vspace{-0.1in}
\item{A compiler and operating system that supports OpenMP code generation,
compiling, linking, and running}

\vspace{-0.1in}
\item{A compiler, operating system, and hardware that supports Cuda/GPU
code generation, compiling, linking, and running}

\end{itemize}

If you problems using the ERT, please read the section on troubleshooting the
ERT.  It contains suggestions that the developers and testers have used when
trying to correct problems running the ERT on new machines with new
configuraiton files.
