On one level, running the ERT is straightforward.  Simply invoke the ``ert''
command with a single command line argument which specifies the file that has
the configuration information.  If all goes well, the tool runs to completion
and at the end tells you the location of resulting the roofline graph file 
and JSON database file.

The ``ert'' command has the following command line options:
\begin{verbatim}
    --version          show program's version number and exit
    -h, --help         show this help message and exit
    --verbose=VERBOSE  Set the verbosity of the screen output
                       [default = 1].
                       = 0 : no output,
                       = 1 : outlines progress,
                       = 2 : good for debugging (prints all
                             commands)
    --quiet            Don't generate any screen output,
                       '--verbose=0'

    Build options:
      --build          Build the micro-kernels [default]
      --no-build       Don't build the micro-kernels
      --build-only     Only build the micro-kernels

    Run options:
      --run            Run the micro-kernels [default]
      --no-run         Don't run the micro-kernels
      --run-only       Only run the micro-kernels

    Post-processing options:
      --post           Run the post-processing [default]
      --no-post        Don't run the post-processing
      --post-only      Only run the post-processing
      --gnuplot        Generate graphs using GNUplot [default]
      --no-gnuplot     Don't generate graphs using GNUplot
\end{verbatim}
The ``verbose''/``quiet'' options allow the user to control the amount of
diagnostic output produced when the ``ert'' command is executed:
\begin{verbatim}
    --verbose=0, -quiet  No diagnostic output is generated

    --verbose=1          An outline is produced as the code
                         is running showing its progress (default)

    --verbose=2          An outline and list of commands being
                         executed are produced as the code
                         is running showing the details of
                         its progress
\end{verbatim}
The ``build''/``no-build''/``build-only'', ``run''/``no-run''/``run-only'', and
``post''/``no-post''/``post-only'' options allow the user to
control what is done during the current execution of ``ert''.  By default, the
micro-kernels are built, run, and the results are post-processed to generate
the final roofline outputs.

It is sometimes necessary to build the micro-kernels on a host machine,
run them on the target machine, and post-process them somewhere else entirely.
This would be accomplished by:
\begin{itemize}

\vspace{-0.1in}
\item{Running ``ert'' on the host machine with the ``-{}-build-only'' option.}

\vspace{-0.1in}
\item{Transferring the entire ERT\_\+RESULT directory created to the target
      machine.}

\vspace{-0.1in}
\item{Running ``ert'' on the target machine with the ``-{}-run-only'' option
      using exactly the same configuration file as on the host.}

\vspace{-0.1in}
\item{Transferring the entire updated ERT\_\+RESULT directory to the
      post-processing machine.}

\vspace{-0.1in}
\item{Running ``ert'' on the target machine with the ``-{}-post-only'' option
      using exactly the same configuration file as on the host.}

\end{itemize}
This is only one example using the build, run, and post-process options.
Other permutations are possible to accommodate needs.

There is also the ``gnuplot/no-gnuplot'' options which can be used to control
if GNUplot is used to produce graphs (in PostScript format) during the
post-processing.  If GNUplot isn't available or you don't want these graphs,
simply use the ``-{}-no-gnuplot'' option.

Of course, this is simply the ``tip of the iceberg''.  What actually happened?
Where did everything get stored and processed?  What do you do if final
results don't look reasonable?  The sections below give more detailed
information about the ERT that can help answer these questions when they
arise.

Also, if you are having trouble getting the ERT to run with your configuration
file use the ``-{}-verbose=2'' option.  This usually gives enough information
to help diagnose the problem.  If not, see the section on troubleshooting the
ERT.

\subsection{Overall Run Structure}

When the ERT is run with a configuration file, e.g.
``ert Config/\+config.\+edison.\+nersc.\+gov.\+01'', it proceeds as
follows:

\begin{enumerate}

\vspace{-0.1in}
\item{Output the version number of the release.}

\vspace{-0.1in}
\item{Read the configuration file.}

\vspace{-0.1in}
\item{Check to see if there is an existing output directory with the same
configuration file.  If so, use that directory.  If not, create a new output
directory and copy the current configuration file there.}

\vspace{-0.1in}
\item{For each value specified by ``ERT\_\+FLOP'':}

\vspace{-0.1in}
\begin{enumerate}

\item{Build the roofline characterization code with that number of FLOPs per
element.}

\item{For each number of MPI processes specified by ``ERT\_\+MPI\_\+PROCS'' and
each number of OpenMP threads specified by ``ERT\_\+OPENMP\_\+THREADS:}

\begin{enumerate}

\item{If the product of the number of MPI processes and OpenMP threads isn't
one of the numbers specified by ``ERT\_\+PROCS\_\+THREADS'' go back to the
previous step.}

\item{If this is a new output directory, run the code as many times as
``ERT\_\+NUM\_\+EXPERIMENTS'' specifies and save all the results.  If not, skip
this step - the code has already been run in this configuration.}

\item{Process the output to produce local results including a variety of
graphs generated by GNUplot in PostScript format.}

\end{enumerate}

\end{enumerate}

\vspace{-0.1in}
\item{Gather all the results to produce the final roofline results as a graph
generated by GNUplot in PostScript format and a JSON output file.  Report the
location of these two files.}

\vspace{-0.1in}
\end{enumerate}

The next sections give more details about these steps and the data they
generate.

\subsection{Storing Results}

When the ERT is run two forms of output are generated.  The obvious one is
all output that goes to the screen by default.  Redirecting it to a file is a
good idea so you can review it later if there is a problem.

In the future, this output will probably go directly to a log file and there
will be a way of controlling the verbosity of the output, especially what goes
directly to the screen.

The other output goes to files in a directory structure below the
``ERT\_\+RESULTS'' directory.  If the ``ERT\_\+RESULTS'' directory doesn't
exist the ERT attempts to create it.  Otherwise, it looks in the directory
for any files of the form ``Run.\#''.  If it finds any, it checks the
configuration file stored under them to see if they match the current
configuration file specified on the command line.

At the moment, this match must be exact in every character.  Eventually, this
will be relaxed as much as possible to detect functional equivalence.

If a match is found, the ERT attempts to continue running and storing results
under the matching ``Run.\#'' directory.  If no match is found, the ERT
creates a new ``Run.\#'' directory where ``\#'' is the smallest positive
integer not currently being used.

Once this directory has been found or created, the current configuration file
is copied there.  This is also the location of the roofline graph file
and the JSON database file when the run completes.

Under the ``Run.\#'', the ERT creates one directory for each of the
``ERT\_\+FLOPS'' values.  These directories are named ``FLOPS.\#'' where
``\#'' is a three digit, zero padded number.  An executable is built in
each of these directories as the value being used changes the code being
compiled.

Under each ``FLOPS.\#'' directory, the ERT creates directories and
subdirectories for type of parallelism turned on in the configuration file.
The nesting of the directories is in the order:  MPI processes, OpenMP
threads, Cuda/GPU thread-blocks, Cuda/GPU threads-per-block.

If any of these forms of parallelism are turned off, subdirectories are not
created for them.  The names of the subdirectories are:  ``MPI.\#'' for the
number of MPI processes, ``OpenMP.\#" for the number of OpenMP threads,
``GPU\_\+Blocks.\#'' for the number of GPU thread-blocks, and
``GPU\_\+Threads.\#'' of the number of GPU threads-per-block.

At this point, the compile time and runtime environment for the code is
completely specified, the code is run, and the results are stored in
appropriate subdirectory.  One file, ``try.\#'', is created for the output
of each experiment.

The format of the ``try.\#'' output will be documented here in future releases
of the ERT.

\subsection{Processed Results}

Next the individual results are
processed to produce individual graphs and summaries.  Then these summaries
are combined to produce the final ERT output for the overall set of runs.

In addition, a significant amount of metadata is carried along through this
process.  This includes timestamps, the ERT configuration file, specific
number for flops/element, MPI processes, OpenMP threads, etc.

In each of subdirectories under each ``FLOPS.\#'' directory containing the
``try.\#'' output (see above), the independent run/experiments are combined
into one result by finding the minimum, median, and maximum bandwidths and
GFLOP rates over the runs and storing this information in a file named
``pre''.

Then the maximum bandwidth and gflop rate for each working set size is
extracted from ``pre'' and stored in ``max''.  Finally, histograms are used to
determine plateaus in the ``max'' data and, along with using the absolute
maximum, the roofline results are determined for all of the experiments.
This result is stored in ``sum''.

The format of all these summary files will be documented here in future
releases of the ERT.

In all cases, the metadata is propagated with the processed data.  Four graphs
are created based on this processed data
unless the ``-{}-no-gnuplot'' option is specified:

\begin{itemize}

\vspace{-0.1in}
\item{``graph1.ps'' - a graph of all the bandwidth data in ``pre'' that shows
run time versus the number of trials for each working set size.  It is
expected that this graph will be set of lines with slope 1.0 as the runtime
should vary linearly with the number of trials once any constant overheads
outside the trial loop are overcome.}

\vspace{-0.1in}
\item{``graph2.ps'' - a graph of all the bandwidth data in ``pre'' that shows
working set size versus gbytes/sec.  The lines connect the result for various
numbers of trails where the working set is constant. The maximum envelope of
this graph is ``graph3.ps''.}

\vspace{-0.1in}
\item{``graph3.ps'' - a graph of maximum of the bandwidth data from ``max''
for each working set size.}

\vspace{-0.1in}
\item{``graph4.ps'' - a graph of maximum of the GFLOPs/sec data from ``max''
for each working set size.}

\end{itemize}

These graphs are useful in diagnosing problems in the overall results of the
ERT.  They are also interesting in their own right as they give detailed
information about the behavior of the architecture and compiler for each
parameter choices, the behavior as a function of work set size, etc.

\subsection{Final Results}
\label{subsection:final_results}

All the individual results are combined into two final results
unless the ``-{}-no-gnuplot'' option is specified then only the JSON output
is generated.
Both are stored directly under the ``Run.\#'' directory:  

\begin{itemize}

\vspace{-0.1in}
\item{``roofline.ps'' - the roofline graph extracted from all the runs made
using the configuration file, ``config.ert'' (also stored in this directory).}

\vspace{-0.1in}
\item{``roofline.json'' - the parameters used to generate the roofline graph
along with all the relevant metadata.  The format of this data needs to be
fully documented but the output is formatted in a way that makes it fairly
easy to see the structural details.  Also, there is separate metadata for the
bandwidth results and the gflop rate results because these, typically, come
from runs with different parameters.  Ideally, the metadata they have in
common should be factored out and placed at a higher level in JSON database
structure with on the differences appearing at the lower levels.}

\end{itemize}

\subsection{Restarting/Continuing Runs}

If the configuration file used is identical, an interrupted/incomplete run
will start from where it left off.  It will repeat all of the building and
post processing but it won't rerun any code that has already run to
completion.

This capability needs to be made more modular and flexible but it provides
some needed functionality even in it's current form.

\subsection{Graphs}

All the graphs generated by the ERT are produced using GNUplot and a GNUplot
script file.
The ``-{}-no-gnuplot'' option disables the production of these graphs and the
associated GNUplot script files.
For a given graph, ``graph.ps'', there will also be a GNUplot
script file, ``graph.gnu'', in the same directory.  The graph can be
regenerated by running GNUplot and typing the command `load ``graph.gnu'''.

If you want to modify any graph produced by the ERT, you can use the GNUplot
script as a starting point and modify it as needed to produce the graph you
would like.  This includes modifying titles and labels, adjusting axes, moving
and adding text, etc.
