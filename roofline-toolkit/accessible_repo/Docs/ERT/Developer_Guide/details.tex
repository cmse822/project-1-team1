\newcommand{\hiddensubsection}[1]{
  \stepcounter{subsection}
  \subsection*{\Alph{chapter}.\arabic{section}.\arabic{subsection}\hspace{1em}{#1}}
} 

This is a colleciton of miscellaneous details gathered from Sam Williams or
found while developing and testing the Roofline Toolkit.

\section{Compiler Flags}

\hiddensubsection{Intel Compiler}
\begin{verbatim}
icc -O3 -fno-alias -fno-fnalias -openmp $ARCH

$ARCH is "-msse3" (Hopper)
         "-xAVX"  (Edison)
         "-mmic"  (Babbage/MIC/Xeon Phi/Knights Corner/KNC/...)
\end{verbatim}

\hiddensubsection{GCC}
\begin{verbatim}
gcc -O3 -fopenmp -fno-strict-aliasing
\end{verbatim}

\hiddensubsection{BlueGeneQ}
\begin{verbatim}
mpixlc_r -O5 -qsmp=omp:noauto ...
\end{verbatim}

\section{Environment Variables}

\hiddensubsection{Linux/Intel Compiler}
\begin{verbatim}
setenv KMP_AFFINITY compact
                    scatter
                    balanced (MIC only)
\end{verbatim}
Also, \verb|KMP_AFFINITY| has verbose and granularity options so you can
see what you're doing and how the hardware multithreading is changing your
view of the machine, for example:
\begin{verbatim}
setenv KMP_AFFINITY=granularity=fine,compact,1,0
\end{verbatim}

\hiddensubsection{Linux/GCC}
\begin{verbatim}
setenv GOMP_CPU_AFFINITY ???
numactl [options] -- command {arguments}
\end{verbatim}

\section{Running}

\hiddensubsection{Hopper}
\begin{verbatim}
aprun -n 4 -d 6 -N 4 -S 1 -ss -cc numa_node <executable>
\end{verbatim}

\hiddensubsection{Edison}
\begin{verbatim}
aprun -n 2 -d 12 -N 2 -S 1 -ss -cc numa_node <executable>
\end{verbatim}

\hiddensubsection{Babbage}
\begin{verbatim}
setenv OMP_NUM_THREADS 240
mpirun.mic -n 1 -host <hostname> -ppn 1 <executable>
\end{verbatim}

\hiddensubsection{BlueGeneQ}
\begin{verbatim}
qsub -t 00:10:00 -n 1 -A BGQtools_esp \
     --proccount 1 \
     --mode c1 \
     --env BG_SHAREDMEMSIZE=32MB:PAMID_VERBOSE=1:\
           BG_COREDUMPDISABLED=1:BG_SMP_FAST_WAKEUP=YES:\
           BG_THREADLAYOUT=1:OMP_PROC_BIND=TRUE:\
           OMP_NUM_THREADS=64:OMP_WAIT_POLICY=active\
     <executable>
\end{verbatim}
You could also run 8 processes per node:
\begin{verbatim}
qsub -t 00:10:00 -n 1 -A BGQtools_esp \
     --proccount 8 \
     --mode c8 \
     --env BG_SHAREDMEMSIZE=32MB:PAMID_VERBOSE=1:\
           BG_COREDUMPDISABLED=1:BG_SMP_FAST_WAKEUP=YES:\
           BG_THREADLAYOUT=2:OMP_PROC_BIND=TRUE:\
           OMP_NUM_THREADS=8:OMP_WAIT_POLICY=active\
     <executable>
\end{verbatim}
Where \verb|-n| is the number of nodes, \verb|--proccount| is the number of
processes, and \verb|--mode| is the number of processes per node.
\begin{verbatim}
setenv BG_THREADLAYOUT 2 <==> setenv KMP_AFFINITY compact
setenv BG_THREADLAYOUT 1 <==> setenv KMP_AFFINITY scatter
\end{verbatim}
They have a balanced option but it's a kludge.

\section{Maximum FLOP Rate}
This is computed a priori with a formula resembling:
\begin{verbatim}
A * CPU-frequency * cores * M

A = 1 (simple CPU/FPU)
    2 (FMA, +/*)

M =  1 (simple CPU)
     2 (SSE)
     4 (AVX/QPX)
     8 (MIC)
    64 (K20 - 192/3)
\end{verbatim}
To achieve this on some (most?) current architecture instruction level
parallelism, ILP, (implicit or explicit) may be needed.  This is necessary to
make sure the internal pipelines feeding the arithmetic units stay filled.
