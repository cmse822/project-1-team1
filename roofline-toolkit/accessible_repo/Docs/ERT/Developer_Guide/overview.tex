The Roofline model of computational performance has been shown to be a useful
simplification of the complexity of processor and memory systems. The goal of
this work is to provide a tool that will allow people writing scientific codes
and libraries to generate the Roofline graph for the computers they are using.

To achieve this an number of tasks need to completed:
\begin{itemize}

\vspace{-0.1in}
\item{Understand the details and nuances of the generation of the Roofline
      model on various computer architectures and operating systems.}

\vspace{-0.1in}
\item{Write the codes that will be used to characterize these
      hardware/software systems.} 

\vspace{-0.1in}
\item{Customize configuration tool(s) to automatically determine
      hardware/software system being tested so that the Roofline codes can be
      compiled and run correctly.}

\vspace{-0.1in}
\item{Produce output in a standard form that can be utilized by other
      performance tools, e.g., Tau.}

\vspace{-0.1in}
\item{Create database(s) to hold the performance information and make them
      accessible to a wide audience.}

\end{itemize}
